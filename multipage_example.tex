\documentclass[a4paper,10pt]{scrartcl}
\usepackage{multipage}
\author{Sakari Cajanus}
\mail{sakari.cajanus@aalto.fi}
\studentnumber{82036R}

\round{Läksy 2}
\course{Solubiosysteemit}{S-115.2500}
\uni{Aalto-yliopisto}
\school{sähkötekniikan korkeakoulu}


\begin{document}
\maketitlepage
\section{Section}
For Bernoulli distribution, i.e. the classes our data points are drawn from the maximum likelihood estimate of the parameter $\hat p_1$ is given by
\begin{align*}
    \hat p_1 = \frac{\sum x_{C=1}}{N}
\end{align*}
The other parameter, $\hat p_2$ is then given by calculation
\begin{align*}
    \hat p_2 = 1 - \hat p_1.
\end{align*}
For the normal distribution (values associated with each class), the MLE's can be calculated using formulas
\begin{align*}
    \hat \mu = \frac{\sum x_{C=i}}{N}
\end{align*}
\begin{align*}
    \hat \sigma^2 = \frac{\sum (x_{C=i}-\hat \mu)^2}{N}.
\end{align*}
In problem 2, we were asked to generate data using a Bernoulli model $p(X=1 |
\theta) = \theta$, $p(X=2 | \theta)=1-\theta$ and then calculate maximum
likelihood estimates for the parameter $\theta$. The source code is shown
in appendix \ref{LiiteB}. Table \ref{tab:bernoulli} shows the results.
\begin{table}[ht]
	\caption{Bernoulli process and MLE's for different sample sizes\vspace{2pt}}
	\label{tab:bernoulli}
	\centering
	\begin{tabular}{rrrr}
		\toprule
		$n$ & $\theta$ & $\hat\theta$ & error \\
		\hline
10        &  0.4000  &  0.6000 & 0.2000 \\
100       &  0.4000  &  0.4200 & 0.0200 \\
1000      &  0.4000  &  0.4090 & 0.0090 \\
10000     &  0.4000  &  0.3971 & 0.0029 \\
		\bottomrule
	\end{tabular}
\end{table}
\clearpage
\appendix
\section{Liite\label{LiiteB}}
\end{document}
