\documentclass[a4paper,10pt]{scrartcl}
\usepackage{onepage}
\author{Sakari Cajanus}
\mail{sakari.cajanus@aalto.fi}
\studentnumber{82036R}

\round{Läksy 2}
\course{Solubiosysteemit}{S-115.2500}
\uni{Aalto-yliopisto}
\school{sähkötekniikan korkeakoulu}


\begin{document}
\pagestyle{fancy}
\begin{center}
\Large\textsc{\theround}\\\normalsize
\end{center}
Daniel E. Koshlandin artikkeli \emph{The Seven Pillars of Life} kuvaa, miten
vaikeaa elämän tunnusmerkkien keksiminen oikeastaan onkaan. Esimerkiksi
perinteiset elämän tunnusmerkkeinä pidetyt käsitykset, kuten lisääntyminen,
eivät välttämättä kestäkään lähempää tarkastelua.

Yhtenä keskeisimmistä elämän tunnusmerkeistä on pidetty lisääntymistä.
Kuitenkaan esimerkiksi tämä päätelmä ei kestä lähempää tarkastelua:
Määritelmään mukaan esimerkiksi yksi (tai "viimeinen") jänis ei ole elossa,
mutta lisääntymiskykyinen jänispari on. Parempi määritelmä elämälle on
mahdollista löytää, kun mietitään asiaa termodynamiikan ja reaktiokinetiikan
kautta: Mitkä ovat selkeästi sellaisia prosesseja, jotka asettavat elämän
erilaiseen asemaan kuin muut prosessit?

Ensimmäiseksi elämän seitsemästä pylväästä Koshland esittää
\textbf{suunnitelmaa}. Suunnitelmalla Koshland tarkoittaa sekä elämän
aineksia, että aineiden vuorovaikutusten kinetiikkaa. Maanpäällisessä
elämässä suunnitelma säilyy ja vaikuttaa nukleiini- ja aminohappojen kautta.
Suunnitelman säilymisen lisäksi sen on myös kyettävä tarvittaessa muuttumaan.
Koshlandin toinen pilari, \textbf{improvisaatio}, tarkoittaa eliöiden kykyä
mukautua tarvittaessa elinympäristönsä muutoksiin.

Seuraava pilari on \textbf{osioituminen}. Ilman tätä kykyä monet reakiot
olisivat mahdottomia: lähes kaikkien reaktioiden onnistuminen tai ainakin
nopeus riippuvat konsentraatioista. Pienetkin eliöt erottautuvat
elinympäristöstään, suuret osioituvat myös rakenteensa sisällä.
\textbf{Energiaa} tarvitaan, koska sekä liike että reaktiot vaativat
energiaa. Entropia kasvaa, mutta auringosta saatavalla energialla eliöt
voivat ylläpitää järjestystä.  \textbf{Uudistuminen} tarkoittaa sekä kudosten
kykyä uusiutua, että eliöiden kykyä "aloittaa alusta" lisääntymisen kautta.
Tällä kuitataan elinaikana kertyneet vahingot.

\textbf{Sopeutuminen} tarkoittaa eliön kykyä muuttaa käytöstään kohdatessaan
ongelmia: Tähän ei sisälly eliön ohjelman muuttuminen (improvisaatio), vaan
sopeutumisen muutokset käytöksessä on kirjattu suunnitelmaa. Seuraava pilari,
\textbf{eristäytyminen}, tarkoittaa eliön kykyä erottaa kemiallisten
reaktioiden reitit toisistaan.

\end{document}
